%%%%%%%%%%%%%%%%%%%%%%%%%%%%%%%%%%%%%%%%%
% "ModernCV" CV and Cover Letter
% LaTeX Template
% Version 1.3 (29/10/16)
%
% This template has been downloaded from:
% http://www.LaTeXTemplates.com
%
% Original author:
% Xavier Danaux (xdanaux@gmail.com) with modifications by:
% Vel (vel@latextemplates.com)
%
% License:
% CC BY-NC-SA 3.0 (http://creativecommons.org/licenses/by-nc-sa/3.0/)
%
% Important note:
% This template requires the moderncv.cls and .sty files to be in the same 
% directory as this .tex file. These files provide the resume style and themes 
% used for structuring the document.
%
%%%%%%%%%%%%%%%%%%%%%%%%%%%%%%%%%%%%%%%%%

%----------------------------------------------------------------------------------------
%	PACKAGES AND OTHER DOCUMENT CONFIGURATIONS
%----------------------------------------------------------------------------------------

\documentclass[11pt,a4paper,sans]{moderncv} % Font sizes: 10, 11, or 12; paper sizes: a4paper, letterpaper, a5paper, legalpaper, executivepaper or landscape; font families: sans or roman

\moderncvstyle{classic} % CV theme - options include: 'casual' (default), 'classic', 'oldstyle' and 'banking'
\moderncvcolor{blue} % CV color - options include: 'blue' (default), 'orange', 'green', 'red', 'purple', 'grey' and 'black'

\usepackage{lipsum} % Used for inserting dummy 'Lorem ipsum' text into the template

\usepackage[scale=0.75]{geometry} % Reduce document margins
%\setlength{\hintscolumnwidth}{3cm} % Uncomment to change the width of the dates column
%\setlength{\makecvtitlenamewidth}{10cm} % For the 'classic' style, uncomment to adjust the width of the space allocated to your name

%----------------------------------------------------------------------------------------
%	NAME AND CONTACT INFORMATION SECTION
%----------------------------------------------------------------------------------------

\firstname{Luís} % Your first name
\familyname{Barbosa} % Your last name

% All information in this block is optional, comment out any lines you don't need
\title{Curriculum Vitae}
%\address{123 Broadway}{City, State 12345}
%\mobile{(000) 111 1111}
%\phone{(000) 111 1112}
%\fax{(000) 111 1113}
\email{luisfernandobarbosa@live.com.pt}
\homepage{luisvilarbarbosa.github.io}{luisvilarbarbosa.github.io} % The first argument is the url for the clickable link, the second argument is the url displayed in the template - this allows special characters to be displayed such as the tilde in this example
%\extrainfo{additional information}
%\photo[70pt][0.4pt]{pictures/picture} % The first bracket is the picture height, the second is the thickness of the frame around the picture (0pt for no frame)
%\quote{"A witty and playful quotation" - John Smith}
\quote{``We all have our time machines.\\Some take us back, they're called memories.\\Some take us forward, they're called dreams.'' - Jeremy Irons}

%----------------------------------------------------------------------------------------

\begin{document}

%----------------------------------------------------------------------------------------
%	COVER LETTER
%----------------------------------------------------------------------------------------

% To remove the cover letter, comment out this entire block

%\clearpage

%\recipient{HR Department}{Corporation\\123 Pleasant Lane\\12345 City, State} % Letter recipient
%\date{\today} % Letter date
%\opening{Dear Sir or Madam,} % Opening greeting
%\closing{Sincerely yours,} % Closing phrase
%\enclosure[Attached]{curriculum vit\ae{}} % List of enclosed documents

%\makelettertitle % Print letter title

%\lipsum[1-2] % Dummy text
%\lipsum[4] % Dummy text

%\makeletterclosing % Print letter signature

%\newpage

%----------------------------------------------------------------------------------------
%	CURRICULUM VITAE
%----------------------------------------------------------------------------------------

\makecvtitle % Print the CV title

%----------------------------------------------------------------------------------------
%	EDUCATION SECTION
%----------------------------------------------------------------------------------------

\section{Education}

\cventry{2014--2019}{Integrated Master in Informatics and Computing Engineering}{FEUP (Faculty of Engineering of the University of Porto)}{Porto}{\textit{GPA: 16 out of 20}}{}

\cventry{2011--2014}{Sciences and Technologies}{Escola Secundária José Régio}{Vila do Conde}{GPA: 17,1 out of 20}{}

\section{Masters Thesis}

\cvitem{Title}{\emph{Intelligent Analysis of Complaints}}
\cvitem{Supervisors}{Professor Henrique Lopes Cardoso \& Ph.D. student Gil Rocha}
\cvitem{Description}{This thesis explored the use of different approaches and features to perform the classification of free text using natural language processing techniques.}
\cvitem{Institution}{LIACC - Artificial Intelligence and Computer Science Laboratory}
\cvitem{GPA}{18 out of 20}

%----------------------------------------------------------------------------------------
%	WORK EXPERIENCE SECTION
%----------------------------------------------------------------------------------------

\section{Experience}

\subsection{Vocational}

\cventry{2019--Present}{Researcher}{\textsc{IA.SAE project}}{Porto}{}{
	The IA.SAE - Inteligência Artificial na Segurança Alimentar e Económica (Artificial Intelligence in Economic and Food Safety) project aims to streamline some of the processes performed inside ASAE - Autoridade de Segurança Alimentar e Económica (Economic and Food Safety Authority).\\
	As a researcher in the project, I have tested multiple machine learning and natural language processing (NLP) techniques and implemented a prototype (``TextCategorizer'') that receives data and returns a predicted label or set of labels, allowing to easily try, test and deploy several text categorization techniques.\\
	The ``TextCategorizer'' project is available on \href{https://github.com/LuisVilarBarbosa/TextCategorizer}{github.com/LuisVilarBarbosa/TextCategorizer}.
}

%------------------------------------------------

\cventry{2018--2019}{Operations Chair}{\textsc{Talk a Bit 2019}}{Porto}{GPA: 17 out of 20}{
	Talk a Bit 2019 was a non-profit tech conference organized by graduating MIEIC (Integrated Master in Informatics and Computing Engineering) and MESW (Master in Software Engineering) students at FEUP, the Faculty of Engineering of the University of Porto, focusing on the future of technology. The conference started in 2013 with less than 150 attendees and emitted 500 tickets in 2019.\\
	As an Operations Chair, I had to manage all the tasks necessary to assure that the conference had the equipment, spaces and personnel necessary to ensure the successful holding of the conference, had to delegate tasks between the members of the Operations department and had to communicate with entities of FEUP and some external entities to ensure some services.\\
	The organization of this conference can be consulted on \href{http://talkabit.org/organization.html}{talkabit.org/organization.html}.
}

%------------------------------------------------

\cventry{2018}{Back-End Developer}{\textsc{FEUP/iTGrow/Critical Software}}{Porto}{GPA: 18 out of 20}{
	As part of the Project Management Laboratory course unit, I worked on the project ``Beyond Sight'', a project proposed and developed for iTGrow/Critical Software that aimed to allow total or partially blind people to follow Microsoft Office PowerPoint presentations by showing the presentation on a website in real-time. Once the presentation is displayed on the website, it is possible for a screen reader to read the text to the user.\\
	The website was developed using React Native and Node.JS.
}

%------------------------------------------------

\cventry{2017--2018}{Back-End Developer}{\textsc{FEUP}}{Porto}{GPA: 17 out of 20}{
	As part of the Software Development Laboratory course unit, I worked on the creation of a website that presented the project ``Stop PropagHate'' and demonstrated how it worked.\\
	``Stop PropagHate'' is an API that aims to bring to the Internet an environment without hate speech using natural language processing techniques.\\
	This project received support from Google's Digital News Initiative.
}

%------------------------------------------------

\cventry{2017}{Profession Week: Engineer}{\textsc{FEUP}}{Porto}{}{
	As part of the Graphical Applications Laboratory course unit, I was later invited to present the developed work on the SPE - ``Semana Profissão: Engenheiro'' (Profession Week: Engineer) which took place on March 27, 28 and 29, 2017.\\
	The invitation was made due to the performance in the course unit and due to the quality of the developed work, having obtained a GPA of 19 out of 20.\\
	The developed work is available on \href{https://github.com/LuisVilarBarbosa/LAIG}{github.com/LuisVilarBarbosa/LAIG} and a working version can be accessed through \href{https://luisvilarbarbosa.github.io}{luisvilarbarbosa.github.io}.
}

%------------------------------------------------

\cventry{2014--2019}{University Projects}{\textsc{FEUP}}{Porto}{}{
	Several projects developed in group, using different languages and technologies:
	\begin{itemize}
		\item Programming Languages: C/C++, C\#, Java, JavaScript, PHP, Prolog, Python.
		\item Programming tools: Microsoft Visual Studio, Eclipse, Git, Atom, Visual Studio Code.
		\item Databases: SQLite, PostgreSQL, SQLServer, MySQL.
		\item Web Technologies: HTML, CSS, Bootstrap, jQuery, WebGL, Node.JS.
		\item Software Testing: Mocha.
		\item Operating Systems: Microsoft Windows, Linux.
		\item Others: .NET, Android, Xamarin, \LaTeX, Kettle.
	\end{itemize}
	Some of these projects are available on \href{https://github.com/LuisVilarBarbosa}{github.com/LuisVilarBarbosa}.
}

%------------------------------------------------

\subsection{Miscellaneous}

\cventry{2019--Present}{}{}{}{}{
	Worked in the ``TextCategorizer'' project which allows to easily try, test and deploy several text categorization techniques.\\
	This project is available on \href{https://github.com/LuisVilarBarbosa/TextCategorizer}{github.com/LuisVilarBarbosa/TextCategorizer}.
}

\cventry{2018--Present}{}{}{}{}{
	Worked in the ``openvpn-server'' project which allows to easily install an OpenVPN server on a Ubuntu/Debian-based machine.\\
	This project is available on \href{https://github.com/LuisVilarBarbosa/openvpn-server}{github.com/LuisVilarBarbosa/openvpn-server}.
}

\cventry{2017}{}{}{}{}{
	Worked in the ``FilesStructureAnalyser'' project which allows performing several types of comparisons and some other operations between files and directories.\\
	This project is available on \href{https://github.com/LuisVilarBarbosa/FilesStructureAnalyser}{github.com/LuisVilarBarbosa/FilesStructureAnalyser}.
}

%----------------------------------------------------------------------------------------
%	AWARDS SECTION
%----------------------------------------------------------------------------------------

\section{Awards}

\cvitem{2016}{Certificate of 2\textsuperscript{nd} place on the Competition ``1\textsuperscript{st} Programming for Optimization and Performance (POP'16)'' organized by the SPeCS laboratory and the IEEE Student Branch of FEUP - Faculdade de Engenharia da Universidade do Porto, Porto.}
\cvitem{2014}{Diploma of student of the Board of Value and Excellence (12\textsuperscript{th} year), Escola Secundária José Régio, Vila do Conde -- GPA: 17.1 out of 20.0.}
\cvitem{2012}{Diploma of student of the Board of Excellence in terms of assiduity (10\textsuperscript{th} year), Escola Secundária José Régio, Vila do Conde.}
\cvitem{2011}{Diploma of student of the Board of Value and Excellence (9\textsuperscript{th} year), Escola Secundária José Régio, Vila do Conde.}
\cvitem{2011}{Certificate of winning team on the 9\textsuperscript{a} Edição do Dia Aberto by FEP - Faculdade de Economia do Porto, Porto.}
\cvitem{2011}{Certificate of participation on the 9\textsuperscript{a} Edição do Dia Aberto by FEP - Faculdade de Economia do Porto, Porto.}
\cvitem{2010}{Diploma of participation on ``Jornadas com Energia'' by EDP Gás.}
\cvitem{2009}{Diploma of student of the Board of Excellence (7\textsuperscript{th} year), EB 2/3 Frei João, Vila do Conde.}
\cvitem{2009}{Diploma of participation in the experimental activities of ``Magia 7'', performed at EB 2/3 Frei João, Vila do Conde.}
\cvitem{2008}{Diploma of student of the Board of Value and Excellence (6\textsuperscript{th} year), EB 2/3 Frei João, Vila do Conde.}
\cvitem{2006}{Diploma in Basic Skills in Information Technologies by Escola Superior de Educação do Instituto Politécnico do Porto.}
\cvitem{2006}{Diploma of participation in the contest ``Uma Aventura... Literária 2006'' promoted by Editorial Caminho.}

%----------------------------------------------------------------------------------------
%	COMPUTER SKILLS SECTION
%----------------------------------------------------------------------------------------

\section{Computer skills}

\cvitem{Basic}{Web Security, Computer Systems Security.}
\cvitem{Intermediate}{JavaScript, Prolog, PHP, HTML, CSS, Node.JS, \LaTeX, OpenOffice, Eclipse, Git, Android, Xamarin, Computer Graphics, Distributed Systems.}
\cvitem{Advanced}{C/C++, SQL, Java, Python, C\#, .NET, Linux, Microsoft Windows, IP Networking.}

%----------------------------------------------------------------------------------------
%	COMMUNICATION SKILLS SECTION
%----------------------------------------------------------------------------------------

%\section{Communication Skills}

%----------------------------------------------------------------------------------------
%	PUBLICATIONS SECTION
%----------------------------------------------------------------------------------------

\section{Publications}

\cvitem{2019}{João Filgueiras, Luís Barbosa, Gil Rocha, Henrique Lopes Cardoso, Luís Paulo Reis, João Pedro Machado, Ana Maria Oliveira (2019). ``Complaint Analysis and Classification for Economic and Food Safety'', in Proceedings of the Second Workshop on Economics and Natural Language Processing, ECONLP 2019, Hong Kong, November 4, 2019, Association for Computational Linguistics, pp. 51–60.}
\cvitem{2019}{L. Barbosa, J. Filgueiras, G. Rocha, H. Lopes Cardoso, L. P. Reis, J. P. Machado, A. C. Caldeira, A. M. Oliveira (2019). ``Automatic Identification of Economic Activities in Complaints'', in Statistical Language and Speech Processing, 7th International Conference, SLSP 2019, Ljubljana, Slovenia, October 14-16, 2019, Springer LNAI 11816, pp. 249-260.}

%----------------------------------------------------------------------------------------
%	LANGUAGES SECTION
%----------------------------------------------------------------------------------------

\section{Languages}

\cvitemwithcomment{Portuguese}{Mothertongue}{}
\cvitemwithcomment{English}{Intermediate}{Conversationally fluent}
\cvitemwithcomment{French}{Basic}{Basic words and phrases only}

%----------------------------------------------------------------------------------------
%	INTERESTS SECTION
%----------------------------------------------------------------------------------------

\section{Interests}

\renewcommand{\listitemsymbol}{-~} % Changes the symbol used for lists

\cvlistdoubleitem{Swimming}{Cycling}
\cvlistdoubleitem{Running}{Driving}

\cvlistdoubleitem{Computer programming}{Movies/Video editing}

\cvlistdoubleitem{Cooking}{Photography}
\cvlistdoubleitem{Singing/Karaoke}{Stand-up comedy}

\cvlistdoubleitem{Listening to music}{Reading}
\cvlistdoubleitem{Do it yourself}{Traveling}

\vfill{}

\begin{center}
	{\scriptsize  Last updated: \today\\
		\href{https://github.com/LuisVilarBarbosa/Curriculum-Vitae}{github.com/LuisVilarBarbosa/Curriculum-Vitae}}
\end{center}

%----------------------------------------------------------------------------------------

\end{document}